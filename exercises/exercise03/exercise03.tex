% Created 2025-06-21 Sat 10:25
% Intended LaTeX compiler: pdflatex
\documentclass[a4paper,11pt]{article}
\usepackage[utf8]{inputenc}
\usepackage[T1]{fontenc}
\usepackage{graphicx}
\usepackage{longtable}
\usepackage{wrapfig}
\usepackage{rotating}
\usepackage[normalem]{ulem}
\usepackage{amsmath}
\usepackage{amssymb}
\usepackage{capt-of}
\usepackage{hyperref}
\usepackage[table]{xcolor}
\usepackage[margin=0.9in,bmargin=1.0in,tmargin=1.0in]{geometry}
\usepackage{algorithm2e}
\usepackage{algorithm}
\usepackage{amsmath}
\usepackage{arydshln}
\usepackage{subcaption}
\newcommand{\point}[1]{\noindent \textbf{#1}}
\usepackage{hyperref}
\usepackage{csquotes}
\usepackage{graphicx}
\usepackage{bm}
\usepackage{subfig}
\usepackage[mla]{ellipsis}
\parindent = 0em
\setlength\parskip{.5\baselineskip}
\usepackage{pgf}
\usepackage{tikz}
\usetikzlibrary{shapes,arrows,automata,quotes}
\usepackage[latin1]{inputenc}
\usepackage{adjustbox}
\author{Prof. Rodrigo Ribeiro}
\date{21-06-2025}
\title{Exercício prático 03\\\medskip
\large BCC328 - Construção de Compiladores I}
\hypersetup{
 pdfauthor={Prof. Rodrigo Ribeiro},
 pdftitle={Exercício prático 03},
 pdfkeywords={},
 pdfsubject={},
 pdfcreator={Emacs 30.1 (Org mode 9.7.11)}, 
 pdflang={English}}
\begin{document}

\maketitle
\section*{Analisador descendente recursivo para a linguagem L1}
\label{sec:orgbd91571}

O objetivo desta atividade é a implementação de um analisador sintático 
descendente recursivo para a linguagem L1
utilizando a biblioteca Megaparsec.
\section*{A Linguagem L1}
\label{sec:org0abfb96}

A linguagem L1 permite a definição de programas simples sem qualquer tipo de desvio de controle.
Programas são apenas uma sequência de comandos. Existem apenas três tipos de comandos em L1:
atribuições, leitura de valores (\textbf{\textbf{read}}) e impressão (\textbf{\textbf{print}}).
\subsection*{Sintaxe da linguagem L1}
\label{sec:orga61f0bc}

A sintaxe da linguagem L1 é definida pela seguinte gramática livre de contexto:

\begin{array}{lcl}
P & \to  & S\, P\:|\:\lambda\\
S & \to  & v := E ; \\
  & \mid & read(E,v);\\
  & \mid & print(E); \\
E & \to  & n \\
  & \mid & v \\
  & \mid & s \\
  & \mid & E + E \\
  & \mid & E - E \\
  & \mid & E * E \\
  & \mid & E \ E \\
\end{array}

A gramática é formada por três variáveis: \(P,\,S\) e \(E\); e pelos seguintes tokens (símbolos do alfabeto):

\begin{itemize}
\item \(v\): representam identificadores. O token de identificador segue as regras usuais presentes em linguagens de programação:
um identitificador começa com uma letra seguida de uma sequência de zero ou mais dígitos ou letras.

\item \(n\): representam constantes numéricas. No momento, vamos suportar apenas números inteiros (tanto positivos, quanto negativos).

\item \(s\): representam literais de strings. A linguagem L1 utiliza aspas duplas para delimitar literais de string.
\end{itemize}
\subsection*{Programa de exemplo}
\label{sec:orgd143d65}

A seguir, apresentamos um programa escrito na linguagem L1:

\begin{verbatim}
x := 0;
read("Digite o valor de x:", x);
print("O valor de x ao quadrado é:" + (x * x));
\end{verbatim}
\section*{Detalhes da entrega}
\label{sec:orgddd5227}

\subsection*{O que deverá ser implementado}
\label{sec:org893ffc1}

Você deverá criar um analisador descendente recursivo utilizando 
a biblioteca Megaparsec. Como resultado, você deverá imprimir a 
árvore de sintaxe produzida por seu analisador, utilizando a função 
\texttt{show} definida para os tipos da árvore sintática.

\begin{verbatim}
recursiveParser :: FilePath -> IO ()
recursiveParser file = error "Not implemented!"
\end{verbatim}

presente no arquivo src/L1/L1.hs, que é o arquivo principal para implementações da linguagem L1. 
A função anterior recebe um nome de arquivo e executa os seguintes passos: 1) lê o conteúdo do arquivo; 2) realizar a análise sintática do conteúdo do arquivo e 3) imprime a árvore de sintaxe produzida.

A implementação da árvore sintática para programas L1 está presente no arquivo \texttt{Syntax.hs}
na pasta L1.Frontend.
\subsection*{Como será feita a entrega}
\label{sec:orgd3ec5a9}

\begin{itemize}
\item As entregas serão feitas utilizando a plataforma Github classroom.
\end{itemize}
\end{document}
