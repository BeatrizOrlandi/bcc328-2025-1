% Created 2025-06-09 Mon 09:07
% Intended LaTeX compiler: pdflatex
\documentclass[a4paper,11pt]{article}
\usepackage[utf8]{inputenc}
\usepackage[T1]{fontenc}
\usepackage{graphicx}
\usepackage{longtable}
\usepackage{wrapfig}
\usepackage{rotating}
\usepackage[normalem]{ulem}
\usepackage{amsmath}
\usepackage{amssymb}
\usepackage{capt-of}
\usepackage{hyperref}
\usepackage[table]{xcolor}
\usepackage[margin=0.9in,bmargin=1.0in,tmargin=1.0in]{geometry}
\usepackage{algorithm2e}
\usepackage{algorithm}
\usepackage{amsmath}
\usepackage{arydshln}
\usepackage{subcaption}
\newcommand{\point}[1]{\noindent \textbf{#1}}
\usepackage{hyperref}
\usepackage{csquotes}
\usepackage{graphicx}
\usepackage{bm}
\usepackage{subfig}
\usepackage[mla]{ellipsis}
\parindent = 0em
\setlength\parskip{.5\baselineskip}
\usepackage{pgf}
\usepackage{tikz}
\usetikzlibrary{shapes,arrows,automata,quotes}
\usepackage[latin1]{inputenc}
\usepackage{adjustbox}
\author{Prof. Rodrigo Ribeiro}
\date{09-06-2025}
\title{Exercício prático 02\\\medskip
\large BCC328 - Construção de Compiladores I}
\hypersetup{
 pdfauthor={Prof. Rodrigo Ribeiro},
 pdftitle={Exercício prático 02},
 pdfkeywords={},
 pdfsubject={},
 pdfcreator={Emacs 29.4 (Org mode 9.7.22)}, 
 pdflang={English}}
\begin{document}

\maketitle
\section*{Analisador léxico para a linguagem L1}
\label{sec:org869963d}

O objetivo desta atividade é a implementação de um analisador léxico para a linguagem L1
utilizando o gerador de analisadores léxicos Alex.
\section*{A Linguagem L1}
\label{sec:org1b1fce4}

A linguagem L1 permite a definição de programas simples sem qualquer tipo de desvio de controle.
Programas são apenas uma sequência de comandos. Existem apenas três tipos de comandos em L1:
atribuições, leitura de valores (\textbf{\textbf{read}}) e impressão (\textbf{\textbf{print}}).
\subsection*{Sintaxe da linguagem L1}
\label{sec:org1746a0e}

A sintaxe da linguagem L1 é definida pela seguinte gramática livre de contexto:

\begin{array}{lcl}
P & \to  & S\, P\:|\:\lambda\\
S & \to  & v := E ; \\
  & \mid & read(E,v);\\
  & \mid & print(E); \\
E & \to  & n \\
  & \mid & v \\
  & \mid & s \\
  & \mid & E + E \\
  & \mid & E - E \\
  & \mid & E * E \\
  & \mid & E \ E \\
\end{array}

A gramática é formada por três variáveis: \(P,\,S\) e \(E\); e pelos seguintes tokens (símbolos do alfabeto):

\begin{itemize}
\item \(v\): representam identificadores. O token de identificador segue as regras usuais presentes em linguagens de programação:
um identitificador começa com uma letra seguida de uma sequência de zero ou mais dígitos ou letras.

\item \(n\): representam constantes numéricas. No momento, vamos suportar apenas números inteiros (tanto positivos, quanto negativos).

\item \(s\): representam literais de strings. A linguagem L1 utiliza aspas duplas para delimitar literais de string.
\end{itemize}
\subsection*{Programa de exemplo}
\label{sec:org379ade7}

A seguir, apresentamos um programa escrito na linguagem L1:

\begin{verbatim}
x := 0;
read("Digite o valor de x:", x);
print("O valor de x ao quadrado é:" + (x * x));
\end{verbatim}
\section*{Detalhes da entrega}
\label{sec:orgbdbe1b2}

\subsection*{O que deverá ser implementado}
\label{sec:orga45591e}

Você deverá criar a especificação de um analisador léxico para a linguagem L1 utilizando a linguagem do gerador de
analisadores léxicos Alex. O descrição do analisador léxico deve estar no arquivo src/L1/Frontend/Lexer.x.

Após a implementação do analisador léxico, você deverá implementar a função

\begin{verbatim}
alexBasedLexer :: FilePath -> IO ()
alexBasedLexer file = error "Not implemtented!"
\end{verbatim}

presente no arquivo src/L1/L1.hs, que é o arquivo principal para implementações da linguagem L1. A função anterior
recebe um nome de arquivo e executa os seguintes passos: 1) lê o conteúdo do arquivo; 2) realizar a análise léxica
do conteúdo do arquivo e 3) imprimir os tokens encontrados, junto com sua respectiva posição de linha e coluna.

A seguir apresentamos parte da saída esperada para o programa L1 de exemplo:

\begin{verbatim}
Identificador x Linha:1 Coluna:1
Atribuição := Linha:1 Coluna:3
Número 0 Linha:1 Coluna:6
Ponto e vírgula ; Linha:1 Coluna:7
Palavra reservada read Linha: 2 Coluna: 1
Parêntesis ( Linha:2 Coluna:5
\end{verbatim}
\subsection*{Como será feita a entrega}
\label{sec:orgcd161f8}

\begin{itemize}
\item As entregas serão feitas utilizando a plataforma Github classroom.

\item \textbf{\textbf{Data limite para a entrega:}} 26/06/2025
\end{itemize}
\end{document}
