% Created 2025-05-12 Mon 08:12
% Intended LaTeX compiler: pdflatex
\documentclass[a4paper,11pt]{article}
\usepackage[utf8]{inputenc}
\usepackage[T1]{fontenc}
\usepackage{graphicx}
\usepackage{longtable}
\usepackage{wrapfig}
\usepackage{rotating}
\usepackage[normalem]{ulem}
\usepackage{amsmath}
\usepackage{amssymb}
\usepackage{capt-of}
\usepackage{hyperref}
\usepackage[table]{xcolor}
\usepackage[margin=0.9in,bmargin=1.0in,tmargin=1.0in]{geometry}
\usepackage{algorithm2e}
\usepackage{algorithm}
\usepackage{amsmath}
\usepackage{arydshln}
\usepackage{subcaption}
\newcommand{\point}[1]{\noindent \textbf{#1}}
\usepackage{hyperref}
\usepackage{csquotes}
\usepackage{graphicx}
\usepackage{bm}
\usepackage{subfig}
\usepackage[mla]{ellipsis}
\parindent = 0em
\setlength\parskip{.5\baselineskip}
\usepackage{pgf}
\usepackage{tikz}
\usetikzlibrary{shapes,arrows,automata,quotes}
\usepackage[latin1]{inputenc}
\usepackage{adjustbox}
\author{Prof. Rodrigo Ribeiro}
\date{12-05-2025}
\title{Exercício prático 01\\\medskip
\large BCC328 - Construção de Compiladores I}
\hypersetup{
 pdfauthor={Prof. Rodrigo Ribeiro},
 pdftitle={Exercício prático 01},
 pdfkeywords={},
 pdfsubject={},
 pdfcreator={Emacs 29.4 (Org mode 9.7.22)}, 
 pdflang={English}}
\begin{document}

\maketitle
\section*{Configuração da ambiente de desenvolvimento}
\label{sec:org9f0b472}


Na disciplina BCC328 - Construção de Compiladores I vamos utilizar um ambiente de desenvolvimento
baseado em \href{https://www.docker.com}{Docker}. A primeira tarefa da disciplina é instalar o Docker, obter o repositório com o
código de exemplo e ser capaz de executá-lo.
\section*{Detalhes da entrega}
\label{sec:org10044f3}

A entrega desta atividade deve ser feita pelo preechimento de um formulário que atesta que o aluno
foi capaz de configurar o ambiente de desenvolvimento.
\subsection*{Data de entrega: 16/05/2025}
\label{sec:org690cce4}
\end{document}
